\documentclass[11pt,a4paper]{jsarticle}
%
\usepackage{amsmath,amssymb}
\usepackage{bm}
\usepackage{graphicx}
\usepackage{ascmac}
%
\setlength{\textwidth}{\fullwidth}
\setlength{\textheight}{40\baselineskip}
\addtolength{\textheight}{\topskip}
\setlength{\voffset}{-0.2in}
\setlength{\topmargin}{0pt}
\setlength{\headheight}{0pt}
\setlength{\headsep}{0pt}
%
\newcommand{\divergence}{\mathrm{div}\,}  %ダイバージェンス
\newcommand{\grad}{\mathrm{grad}\,}  %グラディエント
\newcommand{\rot}{\mathrm{rot}\,}  %ローテーション
%



\title{単語クイズを自動生成して楽しみながら覚えるWebアプリケーションの考案}
\author{k22120 牧野遥斗}
\date{\today}


\begin{document}

\begin{titlepage}
  \begin{center}

      \ \vspace{19mm}

      \LARGE\baselineskip=13mm
        Webプログラミング基礎\\[1mm]
      {\Huge\baselineskip=13mm
      \textbf{単語クイズを自動生成して楽しみながら覚えるWebアプリケーションの考案} \\
      }

      \vspace{80mm}

      \kanjiskip=9pt plus 1pt minus1pt
      \today \\
      K22120\hspace{1zw}牧野遥斗 \\
  \end{center}
\end{titlepage}

% \maketitle
\section{目的}

\section{機能}

\section{目的}

\section{付録}




\end{document}
